

\documentclass{article}

\usepackage{latexsym}
\usepackage{amsmath}
\usepackage{code}

\pagestyle{plain}
\textwidth      150mm
\textheight     210mm
\oddsidemargin  -2mm
\evensidemargin -2mm

\newcommand{\LT}{LT}
\renewcommand{\implies}[0]{\supset}
%\newcommand{\implies}[0]{\rightarrow}
\newcommand{\arrow}[0]{\rightarrow}
\newcommand{\simparrow}[0]{\Longleftrightarrow} 
\newcommand{\proparrow}[0]{\Longrightarrow}
\newcommand{\rightarrowtail}{\longrightarrow}

\newcommand{\ms}[1]{%%\marginpar{\sc ms} 
                   {\bf MS:#1}}


\newtheorem{lemma}{Lemma}

\newtheorem{ex}{Example}
\newenvironment{example}{
        \begin{ex}\rm}%
        {\hfill$\Box$\end{ex}}

\newenvironment{ttline}{\begin{trivlist}\item \tt}{\end{trivlist}}
\newenvironment{ttprog}{\begin{trivlist} %%\small
            \item \tt
        \begin{tabbing}}{\end{tabbing}\end{trivlist}}

\newcommand{\solve}{{\it solve}}
\newcommand{\gd}[0]{~\rule{0.5mm}{2.5mm}~}
\newcommand{\tcons}{\, \vdash_{\scriptsize C} \,}
\newcommand{\tdef}{\, \vdash_{\scriptsize R} \,}

\newcommand{\MRF}{\it MRF}
\newcommand{\ARF}{\it ARF}
\newcommand{\NRF}{\it NRF}


% Martin's macros
   

% groups

%%\newtheorem{lemma}{Lemma}
\newtheorem{definition}{Definition}
%%\newtheorem{theorem}{Theorem}
\newtheorem{corollary}{Corollary}
%%\newtheorem{remark}{Remark}
\newtheorem{conjecture}{Conjecture}
%%\newtheorem{example}{Example}

% arrays

\newcommand{\bi}{\begin{array}[t]{@{}l@{}}}
\newcommand{\ei}{\end{array}}
\newcommand{\ba}{\begin{array}}
\newcommand{\ea}{\end{array}}
\newcommand{\bda}{\[\ba}
\newcommand{\eda}{\ea\]}
\newcommand{\bp}{\begin{quote}\tt\begin{tabbing}}
\newcommand{\ep}{\end{tabbing}\end{quote}}
\newcommand{\set}[1]{\left\{
    \begin{array}{l}#1
    \end{array}
  \right\}}
\newcommand{\sset}[2]{\left\{~#1 \left|
      \begin{array}{l}#2\end{array}
    \right.     \right\}}


% symbols

\newcommand{\good}{good}
 
\newcommand{\err}{{\bf W}}
\newcommand{\VV}{{\cal V}} 
\newcommand{\GT}{{\cal T}_G}
\newcommand{\KK}{{\cal K}}
\newcommand{\TT}{{\cal T}}
\newcommand{\TC}{\mbox{\it TC}}
\newcommand{\UL}{\mbox{UL}}
\newcommand{\sUL}{\mbox{\scriptsize UL}}
\newcommand{\true}{\mbox{\sf true}}
\newcommand{\Int}{\mbox{\it Int}}
\newcommand{\Bool}{\mbox{\it Bool}}
\newcommand{\SF}{\mbox{$\cal S$}}
\newcommand{\tauvec}{\bar{\tau}}
\newcommand{\tvec}{\bar{t}}
\newcommand{\kvec}{\bar{k}}
\newcommand{\xvec}{\bar{x}}
\newcommand{\muvec}{\bar{\mu}}
\newcommand{\alphavec}{\bar{\alpha}}
\newcommand{\betavec}{\bar{\beta}}
\newcommand{\gammavec}{\bar{\gamma}}
\newcommand{\thetavec}{\bar{\theta}}
\newcommand{\deltavec}{\bar{\delta}}
\newcommand{\bvec}{\bar{b}}
\newcommand{\typo}{\Gamma}
\newcommand{\tenv}{\Gamma}
\newcommand{\typoinit}{\typo_0}
\newcommand{\static}{S}
\newcommand{\dynamic}{D}
\newcommand{\sstatic}{{\scriptstyle S}}
\newcommand{\sdynamic}{{\scriptstyle D}}
\newcommand{\BTA}{\mbox{BTA}}
\newcommand{\baseshape}{\bot}
\newcommand{\TCT}{P}
\newcommand{\Haskell}{H98}
\newcommand{\TCTinit}{P_o}

% figures, rules


\newcommand{\tlabel}[1]{\mbox{(#1)}}
\newcommand{\fig}[3]
        {\begin{figure*}[t]#3\
        \caption{\label{#1}#2}\ \hrulefill\ \end{figure*}}
\newcommand{\figurebox}[1]
        {\fbox{\begin{minipage}{\textwidth} #1 \end{minipage}}}
\newcommand{\boxfig}[3]
        {\begin{figure*}\figurebox{#3\caption{\label{#1}#2}}\end{figure*}}
\newcommand{\myirule}[2]{{\renewcommand{\arraystretch}{1.2}\ba{c} #1
                      \\ \hline #2 \ea}}

% relations, operators

\newcommand{\llist}[1]{\langle #1 \rangle}
\newcommand{\sat}{\mbox{\it sat}}
\newcommand{\entail}{\mbox{\it entail}}
\newcommand{\unambig}{\mbox{\it unambig}}
\newcommand{\complete}{\mbox{\it complete}}

\newcommand{\projexclusion}[1]{\mbox{$\bar{\exists}#1$}}
\newcommand{\funinst}{\mbox{\it inst}}
%%\renewcommand{\inst}{\, \vdash^{\scriptstyle i} \,}
%%\newcommand{\inst}{\, \vdash \,}
\newcommand{\ileq}{\preceq}
\newcommand{\ieq}{\simeq}
\newcommand{\topannot}[1]{|#1|}
\newcommand{\gendelta}{\mbox{\it gen}_{\delta}}
\newcommand{\genbeta}{\mbox{\it gen}_{\beta}}
\newcommand{\gen}{\mbox{\it gen}}
\newcommand{\normalize}{\mbox{\it normalize}}
\newcommand{\fail}{\mbox{\it fail}}
\newcommand{\fixpt}{\mbox{${\cal F}$}}
\newcommand{\testpa}{\mbox{${\cal T}$}}
\newcommand{\addpa}{\mbox{${\cal A}$}}
\newcommand{\matchpa}{\mbox{${\cal M}$}}
\newcommand{\refmatchpa}{\mbox{${\cal R}$}}
\newcommand{\wft}[1]{\mbox{\it wft}(#1)}
\newcommand{\tv}{\mbox{\it fv}}
\newcommand{\stv}{\mbox{\scriptsize fv}}
\newcommand{\range}{\mbox{\it range}}
\newcommand{\lub}{\wedge}
\newcommand{\tyrel}[2]{(#1 \preceq #2)}
\newcommand{\eqty}[2]{#1=#2}
%%\newcommand{\ts}{\, \vdash \,}
\newcommand{\turns}{\, \vdash \,}
\newcommand{\tsinst}{\, \vdash^i \,}
\newcommand{\tsttv}{\, \vdash^{ttv} \,}
\newcommand{\tsUL}{\, \vdash_{\mbox{\scriptsize UL}} \,}
\newcommand{\trc}{\, \vdash_{\scriptsize \id{inf}} \,}
\newcommand{\trl}{\, \vdash_{\scriptsize \id{fml}} \,}
\newcommand{\trcp}{\, \vdash_{\scriptstyle inf}' \,}
\newcommand{\genrel}{\, \vdash^{\scriptstyle gen} \,}
\newcommand{\asubty}[2]{(#1 \leq_a #2)}
\newcommand{\ssubty}[2]{(#1 \leq_s #2)}
\newcommand{\fsubty}[2]{(#1 \leq_f #2)}
\newcommand{\subty}[2]{(#1 \leq #2)}
\newcommand{\doubleb}[1]{[\![ #1 ]\!]}
\newcommand{\id}[1]{\mbox{\it #1}}


\newcommand{\mynote}[1]{$\spadesuit${\bf #1}$\clubsuit$}

% spacing

\newcommand{\sgap}{\quad}
\newcommand{\bgap}{\quad\quad}

% reserved words

\newcommand{\mathem}{\sf}
\newcommand{\IN}{\mbox{\mathem in}}
\newcommand{\LET}{\mbox{\mathem let}}
\newcommand{\MLET}{\mbox{\mathem mlet}}
\newcommand{\LETREC}{\mbox{\mathem letrec}}
\newcommand{\FIX}{\mbox{\mathem fix}}
\newcommand{\CASE}{\mbox{\mathem case}}
\newcommand{\TYCASE}{\mbox{\mathem typecase}}
\newcommand{\ELSE}{\mbox{\mathem else}}
\newcommand{\IF}{\mbox{\mathem if}}
\newcommand{\OF}{\mbox{\mathem of}}
\newcommand{\THEN}{\mbox{\mathem then}}
\newcommand{\INST}{\mbox{\mathem instance}}
\newcommand{\OVER}{\mbox{\mathem overload}}
\newcommand{\CLASS}{\mbox{\mathem class}}
\newcommand{\WHERE}{\mbox{\mathem where}}


%%% Local Variables: 
%%% mode: latex
%%% TeX-master: "bta"
%%% End: 


%%%%%%%%%%%%%%%%%%%%%%%%%%%%%%%%%%%%%%%%%%%%%%%%%%%%%%%%%%%%%%%%%%%%%%%%%%%%%%%%

\begin{document}

\title{Description of latest Chameleon Inference Engine}

\maketitle

\section{Type Inference Framework} \label{sec:inf-framework}


\fig{f:type-inference}{Translation to Implication Constraints}{
\bda{cc}
 \tlabel{Var} &
\myirule{(x:\forall\bar{a}.C_2 \Rightarrow t_2) \in \tenv}
        {\tenv, (C_1 \Rightarrow t_1), x \trl (C_1 \wedge C_2 \wedge t_1=t_2 \gd t_2)}
\eda
\bda{cc}
 \ba{cc}
   \tlabel{Abs} &
   \myirule{\mbox{$a$,$b$ fresh} \sgap 
        \tenv\cup\{x:a\},(C_1\wedge t_1=a\arrow b \Rightarrow b), e \trl (F \gd t)}
      {\tenv, (C_1 \Rightarrow t_1), \lambda x.e \trl (F \gd t_1)}
 \ea
 &
 \ba{cc}
   \tlabel{App} &
   \myirule{\mbox{$a$,$b$ fresh} \sgap C_4\equiv C_3\wedge a=b\arrow t_3 
       \\ \tenv, (C_4 \Rightarrow a), e_1 \trl (F_1 \gd t_1) 
       \\   \tenv, (C_4 \Rightarrow b), e_2 \trl (F_2 \gd t_2)}
     {\tenv, (C_3 \Rightarrow t_3), e_1~e_2 \trl (F_1 \wedge F_2 \gd t_3)}
 \ea
\eda
\bda{cc}
 \ba{cc}
   \tlabel{Case} &
   \myirule{\mbox{$a$,$b$ fresh} \sgap \tenv, (C\Rightarrow a), e \trl (F_e \gd t_e)
       \\ \tenv, (C \wedge b=a\arrow t \Rightarrow b), p_i\arrow e_i \trl (F_i \gd t_i)
          \sgap \mbox{for $i\in I$}
       \\ F\equiv F_e \wedge \bigwedge_{i\in I} F_i}
        {\tenv, (C\Rightarrow t), \CASE\ e~ \OF\ [p_i \arrow  e_i]_{i\in I} \trl (F \gd t)}
 \ea
  &
 \ba{cc}
  \tlabel{Pat} &
  \myirule{\mbox{$a$,$b$ fresh} \sgap p \turns \forall \bar{c}.(D \gd \tenv_p \gd t_p)
         \\ \tenv \cup \tenv_p, (C \wedge t=a \arrow b \Rightarrow b), e \trl (F_e \gd t_e)
         \\ F \equiv \forall \bar{c}.(D \implies \bar{\exists}_{\tv(\tenv,\bar{c},t_e)}.F_e)
             \wedge t=t_p \arrow t_e}
      {\tenv, (C \Rightarrow t), p \arrow e \trl (F \gd t)}
 \ea
\eda
\bda{cc}
 \tlabel{Let} &
 \myirule{\mbox{$a$ fresh} \sgap \tenv,(C \Rightarrow a), e_1 \trl (F_1 \gd t_1)
       \\ C_1=\solve(F_1 \gd P) \sgap 
           \bar{a} = \tv(C_1,t_1) - \tv(\tenv) 
       \\ \tenv \cup \{ g:\forall \bar{a}. C_1 \Rightarrow t_1 \}, 
           (C \Rightarrow t), e_2 \trl (F_2 \gd t_2)}
    {\tenv, (C \Rightarrow t), \LET\ g=e_1 ~\IN\ e_2  \trl (F_2 \gd t_2)}
\eda
\bda{cc}
 \tlabel{LetA} &
 \myirule{\bar{a} = \tv(C_1,t_1) \sgap  
        \tenv\cup \{ g : \forall \bar{a}. C_1 \Rightarrow t_1 \},
          (C\wedge C_1 \Rightarrow t_1), e_1 \trl (F_1 \gd t'_1)
      \\  \tenv \cup \{ g:\forall \bar{a}. C_1 \Rightarrow t_1 \}, 
           (C \Rightarrow t), e_2 \trl (F_2 \gd t_2)
      \\ F \equiv F_2 \wedge \forall\bar{a}.(C_1 \implies 
                 \bar{\exists}_{\tv(\tenv,t'_1)}.F_1\wedge t_1=t_1')}
     {\tenv, (C \Rightarrow t), \LET\ 
              \ba{l} g :: C_1 \Rightarrow t_1 \\ g =e_1
                      \ea ~\IN\ e_2 \trl (F \gd t_2)}
\eda
\bda{cc}
 \ba{cc}
   \tlabel{Pat-Var} &
   \myirule{\mbox{$t$ fresh}}
         {x \turns (True \gd \{x:t\} \gd t)}
\ea
&
\ba{cc}
 \tlabel{Pat-K} &
 \myirule{K:\forall\bar{a},\bar{b}.D \Rightarrow t_1 \arrow ... \arrow t_l \arrow T~\bar{a} \sgap
          \bar{b} \cap \bar{a} = \emptyset \\          
          p_k \turns \forall \bar{b_k'}.(D_k' \gd \tenv_{p_k} \gd t_{p_k})  \sgap
          \mbox{$\phi$ m.g.u.~of $t_{p_k}=t_k$} \sgap \mbox{for $k=1,...,l$}
             \sgap {\it dom}(\phi)\cap \bar{b} =\emptyset}
         {K~p_1 ... p_l \turns \forall \bar{b_1'},...,\bar{b_l'} ,\bar{b}.
                 (\phi(D'_1)\wedge ... \phi(D'_l)\wedge \phi(D) \gd 
               \phi(\tenv_{p_1})\cup ... \cup \phi(\tenv_{p_l}) \gd T~\phi(\bar{a}))}
\ea
\eda
}


\bda{llcl}
 \mbox{Constraints} & C & ::= & t=t \mid TC~\bar{t} \mid C \wedge C \\
 \mbox{ImpConstraints} & F & ::= & C \mid \forall \bar{b}. (C \implies \exists\bar{a}.F) \mid F \wedge F 
\eda

\subsection{Inference Algorithm}


The algorithm is formulated in terms of a deduction system
which uses judgments of the form $\tenv, (C \Rightarrow t), e \trl (F \gd t')$
where environment $\tenv$, constraint $C$, type $t$, expression $e$
are input values and implication constraint $F$ and type $t'$ are output values.
The rules are given in Figure~\ref{f:type-inference}.
Component $(C\Rightarrow t)$ represents type information which is provided by
the context. As an invariant we maintain that 
for any valid derivation $\tenv, (C \Rightarrow t), e \trl (F \gd t')$
we have that $F \implies C \wedge t=t'$.


We write $\tenv, e \trl (F \gd t')$ as a short-hand for
$\tenv, (True \Rightarrow a), e \trl (F \gd t')$ where $a$ is a fresh variable.


For patterns we introduce an auxiliary
judgment of the form $p \turns \forall \bar{b}.(D \gd \tenv \gd t)$ 
to compute the set of ``existential'' 
variables $\bar{b}$,
constraint $D$, environment $\tenv$ and type $t$ arising out of pattern $p$. 

We need the ``context'' information to obtain a ``stronger'' inference system.  

\begin{example}
Consider
\begin{code}
class Foo a b where foo :: a->b->Int
instance Foo Int b
f :: Int-> Int
f x = let g' = -- (1)
               (let g :: a->Bool   -- (2)
                    g y = foo x y
                in True)
      in 1
\end{code}
%
At location \tlabel{2}, (roughly) the implication constraint
$forall a.Foo~t_x~a$ arises where $t_x$ is the type of {\tt x}.
At location \tlabel{1}, we need to solve this constraint.
Solving of $forall a.Foo~t_x~a$ is hard unless we take into account
the type information provided by the context (i.e.~the type annotation).
Therefore, in our inference algorithm we push in the constraint that variable {\tt x} has
type {\tt Int}. Then, we can easily solve $forall a.Foo~t_x~a$ if we know
that $t_x=Int$.
\end{example}


\section{Constraint-based Version}



\end{document}
